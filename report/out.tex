\PassOptionsToPackage{unicode=true}{hyperref} % options for packages loaded elsewhere
\PassOptionsToPackage{hyphens}{url}
%
\documentclass[english,a4paper,]{report}
\usepackage{lmodern}
\usepackage{amssymb,amsmath}
\usepackage{ifxetex,ifluatex}
\usepackage{fixltx2e} % provides \textsubscript
\ifnum 0\ifxetex 1\fi\ifluatex 1\fi=0 % if pdftex
  \usepackage[T1]{fontenc}
  \usepackage[utf8]{inputenc}
  \usepackage{textcomp} % provides euro and other symbols
\else % if luatex or xelatex
  \usepackage{unicode-math}
  \defaultfontfeatures{Ligatures=TeX,Scale=MatchLowercase}
\fi
% use upquote if available, for straight quotes in verbatim environments
\IfFileExists{upquote.sty}{\usepackage{upquote}}{}
% use microtype if available
\IfFileExists{microtype.sty}{%
\usepackage[]{microtype}
\UseMicrotypeSet[protrusion]{basicmath} % disable protrusion for tt fonts
}{}
\IfFileExists{parskip.sty}{%
\usepackage{parskip}
}{% else
\setlength{\parindent}{0pt}
\setlength{\parskip}{6pt plus 2pt minus 1pt}
}
\usepackage{hyperref}
\hypersetup{
            pdftitle={Advanced Programming},
            pdfauthor={Ricard Solé Casas},
            pdfborder={0 0 0},
            breaklinks=true}
\urlstyle{same}  % don't use monospace font for urls
\usepackage[margin=1in]{geometry}
% Make links footnotes instead of hotlinks:
\DeclareRobustCommand{\href}[2]{#2\footnote{\url{#1}}}
\setlength{\emergencystretch}{3em}  % prevent overfull lines
\providecommand{\tightlist}{%
  \setlength{\itemsep}{0pt}\setlength{\parskip}{0pt}}
\setcounter{secnumdepth}{5}

% set default figure placement to htbp
\makeatletter
\def\fps@figure{htbp}
\makeatother

\usepackage{minted}
\usemintedstyle{autumn}

\usepackage{fontspec}
\setmonofont{Fira Code}
\defaultfontfeatures{Mapping=tex-text,Scale=MatchLowercase,Ligatures=TeX}
\ifnum 0\ifxetex 1\fi\ifluatex 1\fi=0 % if pdftex
  \usepackage[shorthands=off,main=english]{babel}
\else
  % load polyglossia as late as possible as it *could* call bidi if RTL lang (e.g. Hebrew or Arabic)
  \usepackage{polyglossia}
  \setmainlanguage[]{english}
\fi

\title{Advanced Programming}
\author{Ricard Solé Casas}
\providecommand{\institute}[1]{}
\institute{Google UK \and Ada National College for Digital Skills}
\date{\today}

\begin{document}
\maketitle

\vspace*{\fill}

\section*{Foreword}

The source code for this report and app can be found on
\href{https://github.com/rcsole/zoo-coursework}{Github}. A production
version of the application can be installed from Google's Play Store at
http://bit.ly/ada-zooify. The minimum Android version required is 5.0
(Lollipop).

Some of the decisions taken in building this application will appear to
not follow the suggested guidelines. A detailed explanation is provided
in the \emph{design choices} chapter.

\section*{Declaration}

I confirm that the submitted coursework is my own work and that all
material attributed to others (whether published or unpublished) has
been clearly identified and fully acknowledged and referred to original
sources. I agree that the College has the right to submit my work to the
plagiarism detection service. TurnitinUK for originality checks.

\vspace*{\fill}

{
\setcounter{tocdepth}{2}
\tableofcontents
}
\hypertarget{design-choices}{%
\chapter{Design Choices}\label{design-choices}}

Some of the choices I have made when building this project deviate a
little from the initially suggested approach and can therefore benefit
from some extra explanation.

\hypertarget{android-not-javafx}{%
\section{\texorpdfstring{\href{https://android.com}{Android}, not
\href{https://www.wikiwand.com/en/JavaFX}{JavaFX}}{Android, not JavaFX}}\label{android-not-javafx}}

The design of this assignment seems to be testing more of UI development
than it does OOP and other concepts covered in the \emph{Advanced
Programming} module. While one lecture did brie y touch on what
\emph{JavaFX} is, and how it might be and improvement on Java's
\href{https://www.wikiwand.com/en/Swing_(Java)}{Swing} the lecture was
not nearly in-depth enough to justify using a framework that has been
put on maintenance mode\footnote{Last version is JavaFX 8, released
  nearly 4 years ago in March of 2014.}.

Instead, the more sensible choices seemed to use a web-framework with a
UI built on HTML, CSS, and JavaScript, like
\href{http://sparkjava.com/}{Spark}; or the most popular mobile
platform's SDK, \emph{Android}. I would argue an even better choice for
the given requirements (building a UI) would have been a 100\%
\href{https://www.typescriptlang.org/}{TypeScript} (or plain
\emph{JavaScript}) application using the browser's
\href{http://leveldb.org/}{LevelDB},
\href{https://html.spec.whatwg.org/multipage/webstorage.html}{LocalStorage},
or \href{https://firebase.google.com/}{Firebase} as a persistence layer
but one of the requirements commanded \emph{Java} as the main language
so they were not valid options. My most recent professional experience
in \emph{Java} included building \emph{Android} applications so that
was, from my point view, the most sensible choice given all the
previously stated arguments was to use the \emph{Android Platform}:

\begin{enumerate}
\def\labelenumi{\arabic{enumi}.}
\tightlist
\item
  Existing knowledge
\item
  Industry standards
\item
  Assignment requirements
\end{enumerate}

\hypertarget{pens}{%
\section{Pens}\label{pens}}

The assignment's specification was vague. it was specially so when
describing what a \emph{pen} is and its characteristics. I will unpack
the challenges presented in the specification and explain my
interpretation, in an attempt to get the reader of this report and
myself on the same page:

\begin{center}\rule{0.5\linewidth}{\linethickness}\end{center}

\begin{quote}
\emph{A single pen can only contain animals of the same type.}
\end{quote}

It's unclear what the specification refers to here when it comes to
\emph{type}. An animal type could be one of the provided
\emph{environments} (water, dry, hybrid, and air), or it could also mean
a species (Sloth, Cat, Dog, etc.). For this implementation the
assumption made is that type refers to \emph{environment}, not
\emph{species}. Using the \emph{species} as the type would be a trivial
change with the existing implementation.

\begin{center}\rule{0.5\linewidth}{\linethickness}\end{center}

\begin{quote}
\emph{Each pen has a length, width, and temperature. Aquariums and
aviaries also have a height.}
\end{quote}

There are two things worth noting in this particular part of the
specification:

\begin{enumerate}
\def\labelenumi{\arabic{enumi}.}
\item
  There is no need to store the length, the width, or the height. If a
  pen is dry its are will be land in \(m^{2}\), if it's an aviary it'll
  be \(m^{3}\) in air space, and in water for aquariums. If it's hybrid,
  to accommodate, for example, a Hippo, the pen will have two
  measurements, land in \(m^{2}\), and water in \(m^{3}\).
\item
  The temperature is irrelevant, at least with the examples provided in
  the spec. The person writing the spec did not specify why or how the
  temperature should be used. No animals need specific temperature,
  making that data useless. Were this requirement to change, storing
  temperature data would be a trivial change with this implementation.
\end{enumerate}

\begin{center}\rule{0.5\linewidth}{\linethickness}\end{center}

\begin{quote}
\emph{Pens including water have a water volume and can be either salt or
fresh water.}
\end{quote}

As with to the temperature data whether it is fresh or salt water is
irrelevant. Examples only specify water, dry, petting, air, or part-dry,
party-water. Were this requirement changed, storing water type data
would be a trivial change with this implementation.

\hypertarget{species-class-as-opposed-to-multiple-animal-classes}{%
\section{Species class as opposed to multiple Animal
classes}\label{species-class-as-opposed-to-multiple-animal-classes}}

The assignment presented looks like a more lengthy example of the
canonical concrete \texttt{Dog} class extends from \texttt{abstract}
\texttt{Animal} class. The difference being the need for a UI and some
other entities, like \texttt{Keeper}, and \texttt{Pen}. If that were the
case it would indicate that the coursework designer expected or
suggested we create one \texttt{class} for each sample animal provided.

While that approach works, I believe it would be severely defficient in
a real-world application when managing a zoo. The most obvious
short-coming of the aforementioned architecture is that any new species
registered by the zoo would require actual source code change. Instead I
propose this: treat each animal entity as having a relationship to a
\emph{species} entity. Akin to how in the game of \emph{Pokémon} one
would have a \emph{Pokédex} with all known \emph{Pokémon types}
(\emph{species}) and a separate compartment with all the actual Pokémon.
One wouldn't create a new \texttt{class} for every new \emph{Pokémon
type} that appeared. Instead, one would treat each \emph{species} as a
simply data. This would allow new discoveries about the \emph{species}
to be changed as needed from a UI and without changing the source code
itself.

\hypertarget{keepers-dont-have-a-type}{%
\section{Keepers don't have a type}\label{keepers-dont-have-a-type}}

The module specification makes each of the initial Keepers be
responsible for a given type of pen. I have chosen not to do that. It
seems to be a silly choice. To change that it's a trivial modification.

\hypertarget{overview}{%
\chapter{Overview}\label{overview}}

In this chapter I will attempt to provide a medium-level overview of the
application architecture. I will also mention the few libraries that are
used on top of the already existing \emph{Android Platform} framework.

\hypertarget{libraries}{%
\section{Libraries}\label{libraries}}

\hypertarget{retrofitokhttp}{%
\subsection{Retrofit/OkHttp}\label{retrofitokhttp}}

\begin{quote}
\emph{Type-safe HTTP client for Android and Java by Square, Inc.}

\url{https://github.com/square/retrofit}
\end{quote}

Square's retrofit is built on top of another Square library:
\href{http://square.github.io/okhttp/}{OkHttp}. It comes with a set of
Java annotations that make it straight forward and safe to declare a
remote service in Java and Android. Here it is used to model the
OpenWeatherMap API data.

\hypertarget{lightweight-stream-api}{%
\subsection{Lightweight-Stream-API}\label{lightweight-stream-api}}

\begin{quote}
\emph{Stream API from Java 8 rewritten on iterators for Java 7 and
below.}

\url{https://github.com/aNNiMON/Lightweight-Stream-API}
\end{quote}

To allow for backward compatibility on older Android devices (back to
Lollipop in this application) and still be able to use the nifty
\href{https://docs.oracle.com/javase/8/docs/api/java/util/stream/package-summary.html}{Java
Stream API} I have included a library that backports the functionality
to Android friendly idioms.

\hypertarget{gson}{%
\subsection{Gson}\label{gson}}

\begin{quote}
\emph{Gson is a Java library that can be used to convert Java Objects
into their JSON representation. It can also be used to convert a JSON
string to an equivalent Java object. Gson can work with arbitrary Java
objects including pre-existing objects that you do not have source-code
of.}

\url{https://github.com/google/gson}
\end{quote}

It is widely known that the default \emph{org.json} is a terrible
implementation of a JSON parsing library. It is also widely known the
Gson and Jackson are great implementations solving the same problem. Due
to my familiarity with it I have chosen the former.

I use it to deserialize the Weather data from the API as well as load
the initial seed data from a JSON file.

\hypertarget{guava}{%
\subsection{Guava}\label{guava}}

\begin{quote}
\emph{Guava is a set of core libraries that includes new collection
types (such as multimap and multiset), immutable collections, a graph
library, functional types, an in-memory cache, and APIs/utilities for
concurrency, I/O, hashing, primitives, reflection, string processing,
and much more!}

\url{https://github.com/google/guava}
\end{quote}

I don't think Guava needs much explanation due to its widespread use in
the industry. In the particular case of this application it is used to
have better futures for asynchronous and non-blocking operations, as
well as immutable data structures.

\hypertarget{dagger}{%
\subsection{Dagger}\label{dagger}}

\begin{quote}
\emph{Dagger 2 is a compile-time evolution approach to dependency
injection. Taking the approach started in Dagger 1.x to its ultimate
conclusion, Dagger 2.x eliminates all reflection, and improves code
clarity by removing the traditional ObjectGraph/Injector in favor of
user-specified @Component interfaces.}

\url{https://github.com/google/dagger}
\end{quote}

Like Guava, Dagger 2 is a very common library to use dependency
injection. Even more so when it comes to Android applications, due to
its performance. \emph{Dependency Injection} solves many problems and is
out of the scope of explanation in this report.

\hypertarget{data}{%
\section{Data}\label{data}}

To quote the coursework specification:

\begin{quote}
\emph{Your program should demonstrate good object oriented design and
modelling techniques. \textbf{The domain model should be separate from
user interface (UI)} controllers and views. You are expected to use
object-oriented concepts such as inheritance and polymorphism where it
is required and makes sense to do so.} {[}Emphasis mine{]}
\end{quote}

This section focuses on dicussing the \textbf{bolded} section. The apps
models can be divided many ways, but one that makes sense is local v.
remote, entity v. service. In the context of this application: a service
retrieves or creates data on-demand; an entity is data that will
eventually be persisted.

\hypertarget{local-entities}{%
\subsection{Local entities}\label{local-entities}}

\begin{itemize}
\tightlist
\item
  \texttt{Animal}

  \begin{itemize}
  \tightlist
  \item
    Has a name.
  \item
    Has a Many-to-One relationship to Species entity.
  \end{itemize}
\item
  \texttt{Keeper}

  \begin{itemize}
  \tightlist
  \item
    Has a name.
  \item
    One-to-Many relationship to Pen entity.
  \end{itemize}
\item
  \texttt{Pen}

  \begin{itemize}
  \tightlist
  \item
    Has the amount of land, air, water, and whether or not is pettable.
  \item
    Many-to-One relationship to Keeper entity.
  \end{itemize}
\item
  \texttt{Species}

  \begin{itemize}
  \tightlist
  \item
    Has a name.
  \item
    Has the amount of land, air, water needed, and whether or not is
    pettable.
  \item
    One-to-Many relationship to Animal entity.
  \end{itemize}
\end{itemize}

\hypertarget{local-services}{%
\subsection{Local services}\label{local-services}}

\texttt{AllocatorService} automatically assigns keepers to un-assigned
pens, and un-assigned animals to available pens.

\hypertarget{remote-services}{%
\subsection{Remote services}\label{remote-services}}

\texttt{OpenWeatherMapService} fetches weather data about Barcelona
(where the Zoo is located), and turns it into a POJO (Plain Old Java
Object) using Gson.

\hypertarget{ui}{%
\section{UI}\label{ui}}

To quote the coursework specification:

\begin{quote}
\emph{Your program should demonstrate good object oriented design and
modelling techniques. The domain model should be separate from
\textbf{user interface (UI) controllers and views}. You are expected to
use object-oriented concepts such as inheritance and polymorphism where
it is required and makes sense to do so.} {[}Emphasis mine{]}
\end{quote}

This section focuses on dicussing the \textbf{bolded} section. There are
two main sections to this application: main, where the user can see all
available data; create, where the user can create new data.

For the requirements we read:

\begin{quote}
\emph{If a pen is full or otherwise unable to accommodate the animal,
the user should see an error message explaining why.}
\end{quote}

I believe that an error is always bad if we can avoid it. Precisely for
this reason, when a user wants to assign an animal to a pen the only
choices that they are prompted with are those that can accommodate the
animal.

\hypertarget{critical-evaluation}{%
\chapter{Critical Evaluation}\label{critical-evaluation}}

This application could be improved in the following aspects:

\begin{enumerate}
\def\labelenumi{\arabic{enumi}.}
\tightlist
\item
  Use firebase for the persistance layer.
\item
  Auto-update weather.

  \begin{itemize}
  \tightlist
  \item
    Allow configuration of how often it should auto-update.
  \end{itemize}
\item
  Unit tests.
\item
  Add empty states, when there is no data.
\end{enumerate}

\hypertarget{test-plan}{%
\chapter{Test Plan}\label{test-plan}}

\begin{itemize}
\tightlist
\item
  The program has three main screens:

  \begin{enumerate}
  \def\labelenumi{\arabic{enumi}.}
  \tightlist
  \item
    A list of all animals, displaying key information about each animal.
  \item
    A list of all pens, displaying key information about each pen.
  \item
    A list of staff, displaying key information about each staff member.
  \end{enumerate}
\item
  The user must be able to add new animals

  \begin{itemize}
  \tightlist
  \item
    Add new animal
  \end{itemize}
\item
  Entering all the information regarding animal requirements.

  \begin{itemize}
  \tightlist
  \item
    Add new species
  \end{itemize}
\item
  The user must be able to add new pens, entering all the information
  about the pen.

  \begin{itemize}
  \tightlist
  \item
    Add new pens
  \end{itemize}
\item
  The user must be able to assign animals to pens.

  \begin{itemize}
  \tightlist
  \item
    Assign animal to a pen.
  \end{itemize}
\item
  The user must be able to assign staff to pens.

  \begin{itemize}
  \tightlist
  \item
    Assign a staff member to a pen.
  \end{itemize}
\item
  If a pen is full or otherwise unable to accommodate the animal, the
  user should see an error message explaining why.

  \begin{itemize}
  \tightlist
  \item
    The UI only displays pens that can accomodate the animal.
  \end{itemize}
\item
  If a staff member is assigned to an unsuitable pen, the user should
  see an error message explaining why.

  \begin{itemize}
  \tightlist
  \item
    Decided not to implement this requirement.
  \end{itemize}
\item
  Any animals that have not been assigned a pen should be indicated
  clearly.

  \begin{itemize}
  \tightlist
  \item
    Navigate to animal list with unassigned animals.
  \end{itemize}
\item
  Any pens that have not had staff assigned should generate an alert of
  some kind.

  \begin{itemize}
  \tightlist
  \item
    Navigate to pen list with unassigned animals.
  \end{itemize}
\item
  The program must display the current weather, using data queried from
  \url{https://openweathermap.org/api}

  \begin{itemize}
  \tightlist
  \item
    Highlight weather area.
  \end{itemize}
\item
  The user must be able to refresh the weather data. This should not
  block the UI whilst the request is made.

  \begin{itemize}
  \tightlist
  \item
    Refresh weather data, showing notification.
  \end{itemize}
\item
  Automatic mode, which automatically tries to allocate animals to the
  available pens without input from the user/Automatic mode, which
  automatically tries to allocate staff to the available pens without
  input from the user.

  \begin{itemize}
  \tightlist
  \item
    Create animals and pens, run auto-allocator.
  \end{itemize}
\end{itemize}

\hypertarget{appendix-a-uml-diagram}{%
\chapter{Appendix A: UML Diagram}\label{appendix-a-uml-diagram}}

\end{document}
