\PassOptionsToPackage{unicode=true}{hyperref} % options for packages loaded elsewhere
\PassOptionsToPackage{hyphens}{url}
%
\documentclass[english,a4paper,]{report}
\usepackage{lmodern}
\usepackage{amssymb,amsmath}
\usepackage{ifxetex,ifluatex}
\usepackage{fixltx2e} % provides \textsubscript
\ifnum 0\ifxetex 1\fi\ifluatex 1\fi=0 % if pdftex
  \usepackage[T1]{fontenc}
  \usepackage[utf8]{inputenc}
  \usepackage{textcomp} % provides euro and other symbols
\else % if luatex or xelatex
  \usepackage{unicode-math}
  \defaultfontfeatures{Ligatures=TeX,Scale=MatchLowercase}
\fi
% use upquote if available, for straight quotes in verbatim environments
\IfFileExists{upquote.sty}{\usepackage{upquote}}{}
% use microtype if available
\IfFileExists{microtype.sty}{%
\usepackage[]{microtype}
\UseMicrotypeSet[protrusion]{basicmath} % disable protrusion for tt fonts
}{}
\IfFileExists{parskip.sty}{%
\usepackage{parskip}
}{% else
\setlength{\parindent}{0pt}
\setlength{\parskip}{6pt plus 2pt minus 1pt}
}
\usepackage{hyperref}
\hypersetup{
            pdftitle={Advanced Programming},
            pdfauthor={Ricard Solé Casas},
            pdfborder={0 0 0},
            breaklinks=true}
\urlstyle{same}  % don't use monospace font for urls
\usepackage[margin=1in]{geometry}
% Make links footnotes instead of hotlinks:
\DeclareRobustCommand{\href}[2]{#2\footnote{\url{#1}}}
\setlength{\emergencystretch}{3em}  % prevent overfull lines
\providecommand{\tightlist}{%
  \setlength{\itemsep}{0pt}\setlength{\parskip}{0pt}}
\setcounter{secnumdepth}{5}

% set default figure placement to htbp
\makeatletter
\def\fps@figure{htbp}
\makeatother

\usepackage{minted}
\usemintedstyle{autumn}

\usepackage{fontspec}
\setmonofont{Fira Code}
\defaultfontfeatures{Mapping=tex-text,Scale=MatchLowercase,Ligatures=TeX}
\ifnum 0\ifxetex 1\fi\ifluatex 1\fi=0 % if pdftex
  \usepackage[shorthands=off,main=english]{babel}
\else
  % load polyglossia as late as possible as it *could* call bidi if RTL lang (e.g. Hebrew or Arabic)
  \usepackage{polyglossia}
  \setmainlanguage[]{english}
\fi

\title{Advanced Programming}
\author{Ricard Solé Casas}
\providecommand{\institute}[1]{}
\institute{Google UK \and Ada National College for Digital Skills}
\date{\today}

\begin{document}
\maketitle

\vspace*{\fill}

\section*{Foreword}

The source code for this report and app can be found on
\href{https://github.com/rcsole/zoo-coursework}{Github}. A production
version of the application can be installed from Google's Play Store at
http://bit.ly/ada-zooify. The minimum Android version required is 5.0
(Lollipop).

Some of the decisions taken in building this application will appear to
not follow the suggested guidelines. A detailed explanation is provided
in the \emph{design choices} chapter.

\section*{Declaration}

I confirm that the submitted coursework is my own work and that all
material attributed to others (whether published or unpublished) has
been clearly identified and fully acknowledged and referred to original
sources. I agree that the College has the right to submit my work to the
plagiarism detection service. TurnitinUK for originality checks.

\vspace*{\fill}

{
\setcounter{tocdepth}{2}
\tableofcontents
}
\hypertarget{design-choices}{%
\chapter{Design Choices}\label{design-choices}}

Some of the choice I have made when building this project deviate a
little from the initially suggested approach and can therefore benefit
from some extra explanation.

\hypertarget{android-not-javafx}{%
\section{\texorpdfstring{\href{https://android.com}{Android}, not
\href{https://www.wikiwand.com/en/JavaFX}{JavaFX}}{Android, not JavaFX}}\label{android-not-javafx}}

The design of this assignment seems to be testing more of UI development
than it does OOP and other concepts covered in the \emph{Advanced
Programming} module. While one lecture did briefly touch on what
\emph{JavaFX} is, and how it might be and improvement on Java's
\href{https://www.wikiwand.com/en/Swing_(Java)}{Swing} the lecture was
not nearly in-depth enough to justify using a framework that has been
put on maintenance mode\footnote{Last version is JavaFX 8, released
  nearly 4 years ago in March of 2014.}.

Instead the more sensible choices seemed to use a web-framework with a
UI built on HTML, CSS, and JavaScript, such as
\href{http://sparkjava.com/}{Spark}; or the most popular mobile
platform's SDK, \emph{Android}. I would argue an even better choice for
the given requirements (building a UI) would have been a 100\%
\href{https://www.typescriptlang.org/}{TypeScript} (or plain
\emph{JavaScript}) application using the browser's
\href{http://leveldb.org/}{LevelDB},
\href{https://html.spec.whatwg.org/multipage/webstorage.html}{LocalStorage},
or \href{https://firebase.google.com/}{Firebase} as a persistance layer
but one of the requirements commanded \emph{Java} as the main language
so they were not valid options. My most recient professional experience
in \emph{Java} included building \emph{Android} applications so that
was, from my point view, the most sensible choice given all the
previously stated arguments was to use the \emph{Android Platform}:

\begin{enumerate}
\def\labelenumi{\arabic{enumi}.}
\tightlist
\item
  Existing knowledge
\item
  Industry standards
\item
  Assignment requirements
\end{enumerate}

\hypertarget{pens}{%
\section{Pens}\label{pens}}

The assignment's wording was very vague and it was specially so when
referring to the \emph{pens}. I will unpack the text presented in the
specification and explain my interpretation, in an attempt to get the
reader of this report and myself on the same page:

\begin{center}\rule{0.5\linewidth}{\linethickness}\end{center}

\begin{quote}
\emph{A single pen can only contain animals of the same type.}
\end{quote}

It's unclear what the specification refers to here when it comes to
\emph{type}. An animal type could be one of the provided
\emph{environments} (water, dry, hybrid, and air), or it could also mean
a species (Sloth, Cat, Dog, etc.). For this implementation the
assumption made is that type refers to \emph{environment}, not
\emph{species}. Using the \emph{species} as the type would be a trivial
change with the existing implementation.

\begin{center}\rule{0.5\linewidth}{\linethickness}\end{center}

\begin{quote}
\emph{Each pen has a length, width, and temperature. Aquariums and
aviaries also have a height.}
\end{quote}

There are two things worth noting in this particular part of the
specification:

\begin{enumerate}
\def\labelenumi{\arabic{enumi}.}
\item
  There is no need to store the length, the width, or the height. If a
  pen is dry its are will be land in m\^{}2, if it's an aviary it'll be
  m\^{}3 in air space, and in water for aquariums. If it's hybrid, to be
  able to accomodate, for example, a Hippo, the pen will have two
  measurements, land in m\^{}2, and water in m\^{}3.
\item
  The temperature is irrelevant with the provided specification. None of
  the examples provided by the person writing the spec made mention of
  an animal needing specific temperatures and is therefore irrelevant
  information that need not be stored. Were this requirement changed,
  storing temperature data would be a trivial change with this
  implementation.
\end{enumerate}

\begin{center}\rule{0.5\linewidth}{\linethickness}\end{center}

\begin{quote}
\emph{Pens including water have a water volume and can be either salt or
fresh water.}
\end{quote}

As with to the temperature data whether it is fresh or salt water is
irrelevant. Examples only specify water, dry, petting, air, or part-dry,
party-water. Were this requirement changed, storing water type data
would be a trivial change with this implementation.

\hypertarget{species-class-as-opposed-to-multiple-animal-classes}{%
\section{Species class as opposed to multiple Animal
classes}\label{species-class-as-opposed-to-multiple-animal-classes}}

The assignment presented looks like a more lengthy example of the
canonical concrete \texttt{Dog} class extends from \texttt{abstract}
\texttt{Animal} class. The difference being the need for a UI and some
other entities, like \texttt{Keeper}, and \texttt{Pen}. If that were the
case it would indicate that the coursework designer expected or
suggested we create one \texttt{class} for each sample animal provided.

While that approach works, I believe it would be severely defficient in
a real-world application when managing a zoo. The most obvious
short-coming of the (TODO)afformentioned architecture is that any new
species registered by the zoo would require actual source code change.
Instead I propose this: treat each animal entity as having a
relationship to a \emph{species} entity. Akin to how in the game of
\emph{Pokémon} one would have a \emph{Pokédex} with all known
\emph{Pokémon types} (\emph{species}) and a separate compartment with
all the actual Pokémon. One wouldn't create a new \texttt{class} for
every new \emph{Pokémon type} that appeared. Instead, one would treat
each \emph{species} as a simply data. This would allow new discoveries
about the \emph{species} to be changed as needed from a UI and without
changing the source code itself.

\hypertarget{keepers-dont-have-a-type}{%
\section{Keepers don't have a type}\label{keepers-dont-have-a-type}}

The module specification makes each of the initial Keepers be
responsible for a given type of pen. I have chosen not to do that. It
seems to be a silly choice. To change that it's a trivial modification.

\hypertarget{overview}{%
\chapter{Overview}\label{overview}}

In this chapter I will attempt to provide a medium-level overview of the
application architecture. I will also mention the few libraries that are
used on top of the already existing \emph{Android Platform} framework.

\hypertarget{libraries}{%
\section{Libraries}\label{libraries}}

\hypertarget{retrofitokhttp}{%
\subsection{Retrofit/okHttp}\label{retrofitokhttp}}

\hypertarget{stream-light-api}{%
\subsection{Stream Light API}\label{stream-light-api}}

\hypertarget{gson}{%
\subsection{Gson}\label{gson}}

\hypertarget{guava}{%
\subsection{Guava}\label{guava}}

\hypertarget{dagger}{%
\subsection{Dagger}\label{dagger}}

\hypertarget{dependency-injection}{%
\section{Dependency Injection}\label{dependency-injection}}

From Wikipedia:

\begin{quote}
\end{quote}

\hypertarget{data}{%
\section{Data}\label{data}}

\hypertarget{local-entities}{%
\subsection{Local entities}\label{local-entities}}

\hypertarget{local-services}{%
\subsection{Local services}\label{local-services}}

\hypertarget{remote-services}{%
\subsection{Remote services}\label{remote-services}}

\hypertarget{ui}{%
\section{UI}\label{ui}}

\hypertarget{base-classes}{%
\subsection{Base classes}\label{base-classes}}

\hypertarget{main}{%
\subsection{Main}\label{main}}

\hypertarget{create}{%
\subsection{Create}\label{create}}

\hypertarget{critical-evaluation}{%
\chapter{Critical Evaluation}\label{critical-evaluation}}

\hypertarget{test-plan}{%
\chapter{Test Plan}\label{test-plan}}

\hypertarget{appendix-a-uml-diagram}{%
\chapter{Appendix A: UML Diagram}\label{appendix-a-uml-diagram}}

\end{document}
