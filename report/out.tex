\PassOptionsToPackage{unicode=true}{hyperref} % options for packages loaded elsewhere
\PassOptionsToPackage{hyphens}{url}
%
\documentclass[english,a4paper,]{report}
\usepackage{lmodern}
\usepackage{amssymb,amsmath}
\usepackage{ifxetex,ifluatex}
\usepackage{fixltx2e} % provides \textsubscript
\ifnum 0\ifxetex 1\fi\ifluatex 1\fi=0 % if pdftex
  \usepackage[T1]{fontenc}
  \usepackage[utf8]{inputenc}
  \usepackage{textcomp} % provides euro and other symbols
\else % if luatex or xelatex
  \usepackage{unicode-math}
  \defaultfontfeatures{Ligatures=TeX,Scale=MatchLowercase}
\fi
% use upquote if available, for straight quotes in verbatim environments
\IfFileExists{upquote.sty}{\usepackage{upquote}}{}
% use microtype if available
\IfFileExists{microtype.sty}{%
\usepackage[]{microtype}
\UseMicrotypeSet[protrusion]{basicmath} % disable protrusion for tt fonts
}{}
\IfFileExists{parskip.sty}{%
\usepackage{parskip}
}{% else
\setlength{\parindent}{0pt}
\setlength{\parskip}{6pt plus 2pt minus 1pt}
}
\usepackage{hyperref}
\hypersetup{
            pdftitle={Programming - Part I},
            pdfauthor={Ricard Solé Casas},
            pdfborder={0 0 0},
            breaklinks=true}
\urlstyle{same}  % don't use monospace font for urls
\usepackage[margin=1in]{geometry}
% Make links footnotes instead of hotlinks:
\DeclareRobustCommand{\href}[2]{#2\footnote{\url{#1}}}
\setlength{\emergencystretch}{3em}  % prevent overfull lines
\providecommand{\tightlist}{%
  \setlength{\itemsep}{0pt}\setlength{\parskip}{0pt}}
\setcounter{secnumdepth}{5}

% set default figure placement to htbp
\makeatletter
\def\fps@figure{htbp}
\makeatother

\usepackage{minted}
\usemintedstyle{autumn}

\usepackage{fontspec}
\setmonofont{Fira Code}
\defaultfontfeatures{Mapping=tex-text,Scale=MatchLowercase,Ligatures=TeX}
\ifnum 0\ifxetex 1\fi\ifluatex 1\fi=0 % if pdftex
  \usepackage[shorthands=off,main=english]{babel}
\else
  % load polyglossia as late as possible as it *could* call bidi if RTL lang (e.g. Hebrew or Arabic)
  \usepackage{polyglossia}
  \setmainlanguage[]{english}
\fi

\title{Programming - Part I}
\author{Ricard Solé Casas}
\providecommand{\institute}[1]{}
\institute{Google UK \and Ada National College for Digital Skills}
\date{\today}

\begin{document}
\maketitle

\vspace*{\fill}

\section*{Foreword}

The source code for this report and app can be found on
\href{https://github.com/rcsole/zoo-coursework}{Github}. A production
version of the application can be installed from Google's Play Store at
http://bit.ly/ada-zooify. The minimum Android version required is 5.0
(Lollipop).

Some of the decisions taken in building this app do not follow the
original suggested guidelines. TODO.

\section*{Declaration}

I confirm that the submitted coursework is my own work and that all
material attributed to others (whether published or unpublished) has
been clearly identified and fully acknowledged and referred to original
sources. I agree that the College has the right to submit my work to the
plagiarism detection service. TurnitinUK for originality checks.

\vspace*{\fill}

{
\setcounter{tocdepth}{2}
\tableofcontents
}
\hypertarget{design-choices}{%
\chapter{Design Choices}\label{design-choices}}

I've made some choices for this solution that require some
justification. The project guidelines suggested using the Java
\href{https://www.wikiwand.com/en/Swing_(Java)}{Swing} framework, which
was meant to be replaced by
\href{https://www.wikiwand.com/en/JavaFX}{JavaFX}. However, both
technologies are quite outdated. The former is partially deprecated and
the latter doesn't appear to be well maintained, documentation is
scarce, and Oracle has discontinued support for the editor tools. After
some research I came to the conclusion that, as of June 2017, the
community has pivoted towards using Java for the business logic and
persistence through a web server, while leaving the templating and UI
sections to a much more mature and seasoned technology: the HTML, CSS,
and JavaScript triad.

In this particular project, the server is built on the shoulders of the
Java bindings to the \href{https://playframework.com}{Play} framework
(see footnote for more information). The persistence layer is built on
Java's \href{http://ebean-orm.github.io/}{Ebean} targeting a
\href{https://www.postgresql.org}{PostgreSQL} backend. For deployment, I
used \href{https://heroku.com}{Heroku} and integrated it with
Github\footnote{https://devcenter.heroku.com/articles/github-integration}
for automated deployments.

I've also decided against displaying which questions were answered
incorrectly. The reason for that is that it would serve as
\emph{cheating}. One would just have to retake the entire Quiz and know
which ones they have to actually change. Hiding which answers were
incorrectly answered makes, in my opinion, for a more interesting game
overall.

Likewise I've also decided against prompting the user when skipping a
question. I think that prompting the user \textbf{every time} they want
to skip ahead involves too many clicks.

In the following chapter I will elaborate on the core pieces of the
application.

\end{document}
